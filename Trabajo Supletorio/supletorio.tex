\documentclass{article}
\usepackage[a4paper, total={5.8in, 8in}]{geometry}
\usepackage{hyperref}
\usepackage{amsmath}
\usepackage{graphicx}
\usepackage[utf8]{inputenc}
\usepackage[T1]{fontenc}
\usepackage[spanish]{babel}
\usepackage{mdframed}

\selectlanguage{spanish}
\graphicspath{ {figs/} }

\newmdenv[leftline=false,rightline=false]{topbot}
\newtheorem{pregunta}{Pregunta}

\font\domino=domino
\def\die#1{{\domino#1}}

\DeclareFontFamily{U}{skulls}{}
\DeclareFontShape{U}{skulls}{m}{n}{ <-> skull }{}
\newcommand{\skull}{\text{\usefont{U}{skulls}{m}{n}\symbol{'101}}}

\begin{document}
\begin{center}
  \Large Sueldo y Altura. \\
  % \die1 \die6 \die6 $\skull$ \die6 \die6 \die1 \\
  \null
  Estadística.
\end{center}
\vspace{1cm}

\textbf{Importante:\\
 Escriban el código R para TODAS las preguntas de programación. \\ Cada grupo debe redactar sus resultados.}

\null\hfill

 El objetivo del ejercicio es analizar la relación entre el sueldo anual de una persona y su altura.
\par
Para el ejercicio necesitan leer la tabla sueldo.csv. La tabla tiene cuatro columnas:
\begin{itemize}
  \item Altura en cm.
  \item Peso en libras.
  \item Sueldo en dólares por año.
  \item Sexo: 1 para hombre, 2 para mujer.
\end{itemize}
 \begin{topbot}
   \vspace{0.7em}
   Paso 0 \quad Leer y analizar los datos
   \vspace{0.7em}
 \end{topbot}

\begin{pregunta} (4 puntos)
Lean los datos en R. Hagan un plot de la altura contra el sueldo. Qué pueden concluir al leer el plot?
\par
\textbf{Una recomendación:} dividan la columna sueldo para $1000$ para que los datos sean mas fáciles de leer.
\end{pregunta}

\begin{pregunta} (4 puntos)
Las personas altas tienen sueldos mas altos? Los datos parecen estar ordenados en columnas. Qué interpretación tiene un modelo de regresión?
\end{pregunta}

\begin{topbot}
  \vspace{0.7em}
  Paso 1 \quad Construir el modelo de regresión para explorar la relación.
  \vspace{0.7em}
\end{topbot}

\begin{pregunta} (4 puntos)
Cuál es un buen priemer modelo de regresión para estudiar la relación entre la altura y el sueldo? Por qué?
\end{pregunta}

\begin{pregunta} (4 puntos)
Calculen los coeficientes, $a$ y $b$, del modelo usando lm de R. Interpreten los coeficientes. Qué unidades tienen? Qué significan?
\end{pregunta}

\begin{pregunta} (4 puntos)
 Pueden construir un intervalo de confianza para los coeficientes? Qué representa el intervalo? Para qué sirve el intervalo?
\end{pregunta}

\begin{pregunta} (4 puntos)
 Creen que el sueldo depende solo de la altura? No hay otra variable importante que afecta el sueldo? Qué piensan del sexo?
 \par
 \textbf{Una idea:} Creen una tabla con todas las mujeres y otra con todos los hombres. Luego hagan un modelo de regresión para las mujeres y otro para los hombres. Comparen los coeficientes. Qué peuden concluir?
\end{pregunta}

\end{document}
