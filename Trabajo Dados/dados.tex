\documentclass{article}
\usepackage[a4paper, total={5.2in, 8in}]{geometry}
\usepackage{hyperref}
\usepackage{amsmath}
\usepackage{graphicx}
\usepackage[utf8]{inputenc}
\usepackage[T1]{fontenc}
\usepackage[spanish]{babel}
\usepackage{mdframed}

\selectlanguage{spanish}
\graphicspath{ {figs/} }

\newmdenv[leftline=false,rightline=false]{topbot}
\newtheorem{pregunta}{Pregunta}

\font\domino=domino
\def\die#1{{\domino#1}}

\DeclareFontFamily{U}{skulls}{}
\DeclareFontShape{U}{skulls}{m}{n}{ <-> skull }{}
\newcommand{\skull}{\text{\usefont{U}{skulls}{m}{n}\symbol{'101}}}

\begin{document}
\begin{center}
  \Large Pueden Cargar un Dado Pero no Pueden Alterar una Moneda. \\
  % \die1 \die6 \die6 $\skull$ \die6 \die6 \die1 \\
  \null
  Estadística.
\end{center}
\vspace{1cm}

\textbf{Importante:\\
 Escriban el código R para TODAS las preguntas de programación. \\ Cada grupo debe redactar sus resultados.}

\null\hfill

 El objetivo de todo el ejercicio es tratar de entender cómo y por qué lanzar una moneda es aleatorio.
\par
Van a diseñar y analizar una serie de experimentos para tratar de entender que es una moneda y un dado justos y por qué lanzar una moneda o un dado se consideran experimentos aleatorios.
 \begin{topbot}
   \vspace{0.7em}
   Paso 0 \quad Monedas y dados justos
   \vspace{0.7em}
 \end{topbot}

\begin{pregunta} (4 puntos)
Qué hace que una moneda o un dado sean justos?
\end{pregunta}

\begin{topbot}
  \vspace{0.7em}
  Paso 1 \quad Definir lanzar una moneda y lanzar un dado
  \vspace{0.7em}
\end{topbot}

\begin{pregunta} (4 puntos)
Van a lazar dados y monedas. Para lanzar una moneda:
\begin{enumerate}
  \item La moneda comienza cara abajo.
  \item Lanzar la moneda alto y que de muchas vueltas.
  \item Eje de rotación tiene que ser paralelo al suelo. Sin tambaleos.
  \item Coger la moneda en la palma de la mano mientras está en el aire.
\end{enumerate}
Para lanzar el dado:
\begin{enumerate}
  \item Encuentren una superficie plana.
  \item Dibujen un círculo de mas o menos $60$ cm, en el que van a lanzar el dado.
  \item Batan el dado dentro de un vaso y lancen el dado en el círculo. El lanzamiento tiene que terminar dentro del círculo.
\end{enumerate}

Lancen la moneda $100$ veces y el dado $120$ veces usando sus protocolos. Registren los resultados.

\end{pregunta}

\begin{topbot}
  \vspace{0.7em}
  Paso 2 \quad Modifiquen la moneda y el dado
  \vspace{0.7em}
\end{topbot}

\begin{pregunta} (4 puntos)
Modifiquen la moneda y el dado como quieran para que dejen de ser justos hasta que estén satisfechos. Por jemplo, pueden lijar las esquinas del dado o pueden poner plastilina en una de las caras de la moneda.
\par
Usen el protocolo para lanzar el dado modificado $120$ veces como antes. Lancen la moneda modificada $100$ veces usando el protocolo.
\par
Registren los resultados del dado modificado y la moneda modificada.

Hagan otro experimento: Hagan girar la moneda modificada dentro del círculo y registren si terminó cara arriba o no. Usen un protocolo parecido al del dado: La moneda tiene que terminar dentro del círculo. Repitan el experimento $100$ veces y registren los resultados.

La idea del tercer experimento es la siguiente: Hay dos formas de lanzar una moneda:

\begin{enumerate}
  \item Lanzar la moneda y cogerla en el aire - o dejarla caer en lodo.
  \item Lanzar la moneda, dejarla caer al suelo. Cuando toca el suelo, rebota da algunos giros y para.
\end{enumerate}

El tercer experimento es un modelo de la moneda luego de que toca el suelo y deja de rebotar.

\end{pregunta}

\begin{topbot}
  \vspace{0.7em}
  Paso 3 \quad Primer análisis de los resultados
  \vspace{0.7em}
\end{topbot}

Cómo saben si los resultados son justos? Cuál es la probabilidad de observar sus resultados si asumen que el dado es justo?

\begin{pregunta} (4 puntos)
Cuál es la probabilidad de observar sus resultados si asumen que el dado y la moneda son justas?
\end{pregunta}

\begin{pregunta} (4 puntos)
Qué pasa si repito el experimento? Los resultados van a ser iguales? Parecidos?
\end{pregunta}

\begin{topbot}
  \vspace{0.7em}
  Paso 4 \quad Intervalos de Confianza y tests estadísticos
  \vspace{0.7em}
\end{topbot}

\begin{pregunta} (4 puntos)
Hagan un intervalo de confianza percentíl-bootstrap para verificar si la distribución de la moneda alterada es diferente a una moneda justa. Pueden usar el promedio de los lanzamientos o el número de lanzamientos que son cara o alguna otra estadística. Justifiquen su elección. Qué pueden concluir?
\end{pregunta}

\begin{pregunta} (4 puntos)
  Para los dados hagan un test de permutaciones. Qué pueden concluir?
\end{pregunta}


\end{document}
