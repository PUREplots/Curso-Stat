\documentclass{article}
\usepackage[a4paper, total={5.8in, 8in}]{geometry}
\usepackage{hyperref}
\usepackage{amsmath}
\usepackage{graphicx}
\usepackage[utf8]{inputenc}
\usepackage[T1]{fontenc}
\usepackage[spanish]{babel}
\usepackage{mdframed}

\selectlanguage{spanish}
\graphicspath{ {figs/} }

\newmdenv[leftline=false,rightline=false]{topbot}
\newtheorem{pregunta}{Pregunta}

\font\domino=domino
\def\die#1{{\domino#1}}

\DeclareFontFamily{U}{skulls}{}
\DeclareFontShape{U}{skulls}{m}{n}{ <-> skull }{}
\newcommand{\skull}{\text{\usefont{U}{skulls}{m}{n}\symbol{'101}}}

\begin{document}
\begin{center}
  \Large Regresión lineal. \\
  % \die1 \die6 \die6 $\skull$ \die6 \die6 \die1 \\
  \null
  Estadística.
\end{center}
\vspace{1cm}

\textbf{Importante:\\
 Escriban el código R para TODAS las preguntas de programación. \\ Cada grupo debe redactar sus resultados.}

\null\hfill

 El objetivo del ejercicio es analizar la relación entre dos variables.
\par
Van a registrar y analizar datos de dos experimentos para deducir algunas relaciones importantes en física.
 \begin{topbot}
   \vspace{0.7em}
   Paso 0 \quad Registrar y analizar los datos
   \vspace{0.7em}
 \end{topbot}

\begin{pregunta} (4 puntos)
Registren y lean los datos en R. Hagan un plot de las observaciones para cada experimento. Qué pueden concluir al leer los plots?
\end{pregunta}

\begin{pregunta} (4 puntos)
Las relaciones son lineales o necesitan hacer algún tipo de transformción en las variables? Hay alguna intuición física que les pueda ayudar para descubrir las tranformaciones?
\end{pregunta}

\begin{topbot}
  \vspace{0.7em}
  Paso 1 \quad Construir el modelo de regresión
  \vspace{0.7em}
\end{topbot}

\begin{pregunta} (4 puntos)
Cuál es un buen un modelo de regresión para cada una de las relaciones? Por qué?
\end{pregunta}

\begin{pregunta} (4 puntos)
Calculen los coeficientes, $a$ y $b$, del modelo usando lm de R. Interpreten los coeficientes. Qué unidades tienen? Qué significan? Pueden construir un intervalo de confianza para los coeficientes? Qué representa el intervalo?
\end{pregunta}

\begin{pregunta} (4 puntos)
Qué limitaciones tienen los modelos que usaron? Pueden proponer alguna alternativa? Cuándo son útiles los modelos?
\end{pregunta}

\end{document}
